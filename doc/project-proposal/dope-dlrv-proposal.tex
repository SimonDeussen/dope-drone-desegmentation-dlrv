\documentclass[rnd]{mas_proposal}
\usepackage[utf8]{inputenc}
\usepackage{amsmath}
\usepackage{amsfonts}
\usepackage{amssymb}
\usepackage{graphicx}

\title{Semantic Segmentation of aerial images of forest scenery}
\author{Malika Navaratna, Urvashi Negi, Zain Ul Haq, Simon Deussen}
\supervisors{Prof. Sebastian Houben}
\date{November 2021}

% \thirdpartylogo{path/to/your/image}

\begin{document}

\maketitle

\pagestyle{plain}

\section{Introduction}

    The objective of the project is to perform semantic segmentation using Deep learning techniques. The project will use
    labelled data of forest scenery captured from an aerial view most commonly by unmanned aerial vehicles. An Example of the
    the data that will be used for the project is given in the Figure\ref{fig:Example image}. 
    \begin{figure}[h!]
        \centering
        \includegraphics[width=10cm]{"images/forest_image_1.JPG"}
        \caption{Forest imagery}
        \label{fig:Example image}
        \end{figure} 
    \linebreak
    The expected final result is to have a pixel-wise segmentation of the forest into the following categories    
        \begin{itemize}
            \item Empty forest area
            \item Trees
            \item Shrubbery
            \item Logs(cut down stacks of trees)
            \item Roads(dirt and tarmac)
        \end{itemize}

\subsection{Project Plan}

    \subsection{Models and tools}

    \subsection{Data sets}

    \subsection{Potential problems}
    


\section{Related Work}
\begin{itemize}
    \item What have other people done?
    \item Why is it not sufficient?
\end{itemize}




\nocite{*}

\bibliographystyle{plainnat} % Use the plainnat bibliography style
\bibliography{bibliography.bib} % Use the bibliography.bib file as the source of references




\end{document}
